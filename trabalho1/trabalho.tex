\documentclass[
    12pt, % tamanho da fonte
    a4paper, % tamanho do papel.
    oneside, % impede a configuração para impressão frente e verso
    english, % idioma adicional para hifenização
    brazil % o último idioma é o principal do documento
]{abntex2}

% ---
% Pacotes básicos
% ---
\usepackage{lmodern}
\usepackage[T1]{fontenc} % Selecao de codigos de fonte.
\usepackage[utf8]{inputenc} % Codificacao do documento (conversão automática dos acentos)
\usepackage{csquotes}
\usepackage{indentfirst} % Indenta o primeiro parágrafo de cada seção.
\usepackage{color} % Controle das cores
\usepackage{graphicx} % Inclusão de gráficos
\usepackage{microtype} % para melhorias de justificação
% ---

% ---
% Pacotes de citações
% ---
\usepackage[style=abnt]{biblatex}
\addbibresource{referencias.bib}

% ---
% Informações de dados para CAPA e FOLHA DE ROSTO
% ---
\title{Informática para usuários com deficiência}
\author{Bruno Mello\and Geraldo Fada\and Luciano Lage\and Rodrigo Correa}
\local{Brasil, Rio de Janeiro}
\date{\today}
\orientador[Professor:]{José Raphael Bokehi}
\instituicao{Universidade Federal Fluminense}
\tipotrabalho{Trabalho Acadêmico}

% ---
% Configurações de aparência do PDF final
% ---
% alterando o aspecto da cor azul
\definecolor{blue}{RGB}{60,100,120}
% informações do PDF
\makeatletter
\hypersetup{
    %pagebackref=true,
    pdftitle={\@title},
    pdfauthor={\@author},
    pdfsubject={\imprimirpreambulo},
    pdfcreator={LaTeX with abnTeX2},
    pdfkeywords={abnt}{latex}{abntex}{abntex2}{trabalho acadêmico},
    colorlinks=true, % false: boxed links; true: colored links
    linkcolor=blue, % color of internal links
    citecolor=blue, % color of links to bibliography
    filecolor=magenta, % color of file links
    urlcolor=blue,
    bookmarksdepth=4
}
\makeatother
% ---

% ---
% Espaçamentos entre linhas e parágrafos
% ---
% O tamanho do parágrafo é dado por:
\setlength{\parindent}{1.3cm}
% Controle do espaçamento entre um parágrafo e outro:
\setlength{\parskip}{0.2cm}  % tente também \onelineskip

% ----
% Início do documento
% ----
\begin{document}

% Seleciona o idioma do documento (conforme pacotes do babel)
\selectlanguage{brazil}

% Retira espaço extra obsoleto entre as frases.
\frenchspacing

\imprimirfolhaderosto

% ---
% inserir o sumario
% ---
\pdfbookmark[0]{\contentsname}{toc}
\tableofcontents*
\cleardoublepage

\textual
\chapter{Introdução}

No dia a dia de toda a população, a tecnologia está cada vez mais inserida,
seja através de computadores, notebooks, smartphones, tablets, entre outros.
Diante desse cenário do avanço da tecnologia em nossas vidas, cada vez mais
se faz necessário ampliarmos o nosso processo de inclusão digital. Tarefas
corriqueiras para uns e outros como ligar um computador, digitar um texto
ou acessar a internet, muitas vezes acabam sendo um universo completamente
desconhecido para pessoas com deficiência.

De acordo com a organização mundial de saúde, estima-se que mais de um bilhão
de pessoas vivem com algum tipo de deficiência \cite{Disability}. Pessoas com algum
tipo de deficiência estão mais propensas a sofrerem mais problemas de saúde,
têm menos acesso à educação, consequentemente menos oportunidades de trabalho
sendo assim mais provável elas estarem na linha de pobreza do que quando
comparamos a pessoas sem deficiência.

O tema da acessibilidade é muito levantado enquanto sociedade, porém muitas
vezes é desconhecido o seu significado. De acordo com o dicionário, acessibilidade
é “propriedade do material confeccionado para que qualquer pessoa tenha acesso,
consiga ver, usar, compreender; diz-se, principalmente, do material que se
destina à inclusão social de pessoas com alguma deficiência" \cite{Acessibilidade}.
    
Logo é fácil ver que o acesso a informática deve ser algo estudado, tendo em
vista que o uso da tecnologia é imprescindível para todos, seja no momento de
lazer, acessando as redes sociais, vendo séries e filmes, como no âmbito profissional,
desde trabalhando com um excel até mesmo na programação.


\chapter{Desenvolvimento}
\section{Deficiência visual}
Nos dias de hoje a tecnologia voltada para a inclusão do deficiente visual avança
de maneira mais rápida e dinâmica do que era anos atrás, ao parar e pesquisar sobre
o assunto, podemos achar tecnologias como impressoras em braile, computadores feito
para cegos, bengalas com sensores, celulares e dispositivos adaptados, programas
com tecnologia adaptativa e assistiva, entre outros, todos itens que tem por
intuito incluir e auxiliar o deficiente visual nas tarefas do dia a dia, nos
estudos, trabalho e entretenimento. Filtrando um pouco para o mercado brasileiro,
programas como o DOSVOX, VIRTUAL VISION e o JAWS, possibilitaram dessas pessoas
o universo da informática.
 
Conforme entrevista do professor de informática Valter Júnior a TV Senado em
meados de 2018 \cite{tvsenadoInformaticaAuxiliaDia}, a pessoa deficiente visual total ou parcial,
seja ela de nascença ou desenvolvida ao longo de vida, tem um leque de opções
para se capacitar, inteirar, adaptar e recuperar o acesso às tecnologias,
o melhor de tudo é sem precisar de grandes investimentos ou buscas elaboradas.
Como dito por Valter, os próprios deficientes visuais estão se capacitando
e tornando-se multiplicadores de tecnologia que tem tanto para os beneficiar.
É possível encontrar cursos gratuitos no Youtube, ou até mesmo em instituições
particulares, como o Bradesco, ou públicas como o Instituto Benjamin Constant \cite{InformaticaParaPessoas,ProgramasInformaticaNa}.
 
O professor Valter Júnior faz um belo trabalho com o canal no Youtube 
``Projeto João 9'' \cite{ProjetoJoaoYouTube}, em que disponibiliza áudio livros
e ministra aulas gratuitas de informática com intuito de capacitar as pessoas com deficiência visual para o
mercado de trabalho, Valter dá aulas desde noções básicas de Windows através
de softwares de assistência visual, até cursos mais completos do pacote Office,
utilizando os mesmo softwares. Assim como o professor Valter, outro nome
conhecido nesse meio é o do desenvolvedor Lucas Radaelli \cite{LucasRadaelliYouTube,infoComoUmDeficiente},
que através do canal
que leva seu próprio nome, ele mostra para o mundo o ponto de vista de um cego
utilizando a tecnologia, e fazendo tudo o que qualquer pessoa pode fazer.
 
Abrangendo um pouco sobre as tecnologias voltadas para auxiliar esse público,
pode-se mencionar um software, DOSVOX \cite{ProjetoDOSVOX}, e um hardware,
o teclado para deficientes visuais \cite{TecladoParaDeficientes2017}. Em 1993, o Núcleo de Computação Eletrônica
(NCE) do centro de Ciências Matemáticas e da Natureza da Universidade Federal
do Rio de Janeiro (UFRJ), criou o sistema chamado DOSVOX, que tem por objetivo
auxiliar deficientes visuais com o uso do computador. O DOSVOX possibilita os
seus usuários uma gama de tarefas como edição de texto, sendo possível realizar
uma impressão em braile ou comum, leitura de textos anteriormente transcritos
por meio da audição, acesso por maneira falada, ferramentas voltadas para produtividade:
como calculadora, agenda, etc; assim como a possibilidade de acessar diversos
jogos. O sistema DOSVOX desde sua criação vem se desenvolvendo e aprimorando,
para continuar auxiliando todo um público em potencial.
 
Já na parte de hardwares, existem diversos tipos de teclado para deficientes visuais,
que podem ser adaptados dependendo da necessidade do usuário, seja ele invisual total.
Os teclados comuns possuem dois traços nas letras “F” e “J” que tem por intuito
indicar o posicionamento das teclas, porém nem sempre são totalmente corretos,
visto a quantidade de variações de layout de teclados dependendo do padrão adotado.
Já os teclados com sistema braile são uma boa opção para aqueles usuários que já estão
familiarizados com o sistema, pois facilitam na hora de identificar o posicionamento
das teclas, esse tipo de teclado é mais indicado para aqueles usuários que possuem
perda total da visão. Em contra partida, para usuários que possuem dificuldade de
enxergar ou baixa visão, existem os teclados com cores destacadas ou com caracteres
de tamanho ampliado, que facilitam na hora da digitação do usuário.

\section{Deficiência auditiva}
Outro grupo de usuários que também pode desfrutar de uma melhoria na acessibilidade
da informática são os deficientes auditivos parciais ou totais. Nesses casos a
dificuldade de acesso se dá mais frequentemente no uso e consumo  de mídias que
requerem um maior nível de interação, como por exemplo jogos digitais, vídeos, séries e filmes.

Pensando no contexto de jogos digitais, existem algumas especificidades que precisam
ser atacadas para que o usuário deficiente auditivo consiga desfrutar da mesma
experiência de um usuário sem tais barreiras. A revista NerdWeek fez uma entrevista
\cite{ConfiraJogosCom2019} excelente com o Felipe de Castro, um estudante de Jogos Digitais
que possui surdez profunda. Um exemplo de jogo que infelizmente não pode ser
considerado acessível nesse ponto é a franquia Call of Duty, especialmente quando falamos
da modalidade multijogador destes jogos. Um jogador pode e deve se atentar ao áudio
dos passos de um jogador inimigo para identificar o posicionamento do mesmo no mapa,
e não ter acesso à essa informação gera uma desvantagem considerável. No entanto essa
falta de acessibilidade poderia ser contornada caso a Activision, empresa responsável
pelo desenvolvimento da saga, implementasse uma feature de representação visual de
áudio. Essa funcionalidade poderia ser algo como um ícone no canto da tela que indicasse
a direção de onde veio o som.

Já no caso do game “The Legend of Zelda: Breath of the Wild” a Nintendo, empresa responsável
pelo desenvolvimento do jogo, fez um excelente trabalho ao garantir que todas as features
do jogo fossem inclusivas nesse aspecto. Podemos pegar como exemplo uma feature do jogo
referente à procura de templos e itens escondidos pelo mapa. Quando Link (protagonista da
saga Zelda) se aproxima desses pontos de interesse existe um indicador visual que brilha
e vibra, além de uma sinalização sonora. Também existe um outro indicador que representa a
quantidade de barulho que o jogador está fazendo ao se movimentar, muito útil para momentos
em que o mesmo segue por uma abordagem furtiva. Além disso, quando Link está com muito frio
existe um ícone de uma nuvem, indicando que o jogador deve procurar abrigo.

Em síntese, podemos perceber que existem algumas maneiras de contornar as limitações mais
óbvias, porém a mentalidade de “contornar o problema” não é ideal e deve ser evitada. Idealmente
espera-se que as empresas responsáveis implementem em seu fluxo de produção e testes uma etapa
de testes de acessibilidade, que consistiria na contratação de pessoas com deficiência auditiva
parcial e total para jogar e testar o produto final. Somente dessa forma seria possível ter
certeza que o produto é acessível para tal público.

Já no contexto de mídias como vídeos, séries e filmes, onde grande parte do conteúdo relevante
se dá através do áudio como diálogos, onomatopéias alguns sons ambientes, é essencial que
esse conteúdo se mantenha o mais acessível possível. Uma boa forma de garantir isso é através
da exibição de legendas. Grandes plataformas de streaming de conteúdo como a Netflix, Amazon
Prime etc possuem uma feature de legendas com descrição do áudio (para os casos de onomatopéias
e sons ambiente) que passam por uma curadoria específica antes que o conteúdo vá ao ar. No
entanto, plataformas como o youtube, que são muito orientadas a conteúdo produzido pelo próprio
usuário da plataforma, tendem a pecar nessa questão. No caso desta plataforma a responsabilidade
de produzir uma legenda para acompanhar o vídeo enviado fica a cargo do próprio produtor do
conteúdo, que grande parte das vezes não o faz, seja não julgar necessário ou por não estar
disposto a investir tempo ou dinheiro em um esforço de legendar essas produções.

Para mitigar esse dilema o Youtube possui uma ferramenta \cite{UsarTranscricaoAutomatica} de transcrição
automática de áudio em texto que funciona através do reconhecimento de fala e aprendizado de
máquina. Além disso, alguns produtores de conteúdo como o Tom Scott \cite{TomScottYouTube}
permitem que o próprio público envie correções e novas traduções das legendas de seus vídeos.

\section{Deficiência mental}
Diferente dos outros grupos de deficiência, a mental não possui grande
alcance no quesito de acessibilidade na informática. Pode-se dizer que
esse caso é contrário ao que já foi abordado no trabalho, pois o próprio
campo da infórmatica tende a ajudar esse grupo ao invés de se moldar para
ser inclusivo.

Utilizando como exemplo as deficiências auditivas e visuais, ambas possuem
integrações para acessibilidade com o propósito de incluir o usuário ao sistema.
Isso não é comum para casos mentais, não existem produtos que facilitem o acesso
para esse grupo, porém isso é comum em aplicativos,
sites e jogos com opções de inclusão como por exemplo um modo de daltonismo.

A web é um grande exemplo disso, com tag específicas no padrão HTML
para deficientes visuais que facilitam softwares de descrição para cegos
fazer a leitura da página. Ou jogos famosos como Fortnite que incluem descrição
de áudios para pessoas surdas conseguirem se adequar ao jogo \cite{morganDeafFortnitePlayer2019}.

A infórmatica está mais relacionada com o desenvolvimento e facilitação do dia a dia dos deficientes mentais.
Temos casos de softwares feitos especificamentes para ajudar
no ensino de crianças com autismo e síndrome de down \cite{pombal2009papel};
um teclado da marca Tix \cite{TecladoInteligenteTiX} que tem a proposta de ser
altamente inclusivo e possui cases de sucesso com jovens deficientes, por exemplo o Daniel Antônio \cite{DanielAntonioConceicao}
que possui a fala e os movimentos do corpo comprometidos pela paralisia cerebral e
descobriu uma nova forma de se comunicar.

Além de marcas de nicho, empresas gigantes como a Microsoft também
estão contribuindo no cenário. Em 2018 na E3 (Electronic Entertainment Expo),
o evento anual de maior importância para jogos eletrônicos \cite{wattsE32019Video,yeoSonySkippingOut2018},
foi quando a Microsoft anunciou seu controle adaptativo \cite{xboxIntroducingXboxAdaptive}.
Mais uma forma inclusiva do usuário interagir com o computador, porém nesse caso
foi voltado para a indústria de \textit{video-game}.

A relevância de um controle inclusivo para jogos digitais pode ser percebida
quando existem estudos comprovando a eficiência dos jogos digitais para facilitar
a aprendizagem de pessoas com deficiências intelectuais \cite{aimi2015jogos}
ou até mesmo para reabilitação cognitiva de distúrbios ezquifrônicos \cite{menezes2017aplicaccao}.

Por esses pontos é fácil perceber uma mudança na ideia de acessibilidade propriamente dita
para deficiente mentais, é mais comum observar a tecnologia como um apoio para essas
pessoas na sua vida do que ver softwares incluindo-os como um público-alvo do seu produto.
Provavelmente por ser um problema de natureza completamente diferente e com soluções não tão
simples. Não é possível criar uma opção num software em que a pessoa com deficiencia intelectual possa
simplemente ativar para poder usa-lo de forma melhor, como ocorre para
casos de surdez que basta descritivos visuais para integrar esse grupo.

\section{Deficiência motora}
O ser humano interage com o computador por meio dos dispositivos de entrada e saída, e
essa interação muitas vezes é dificultada devido há uma falta de recurso para certas pessoas
com deficiências motoras, pois a utilização desses periféricos pode não ser trivial. Logo,
por meio da tecnologia e informática, se pode criar recursos adaptados para pessoas com
deficiência.

Assim como dito em outras partes deste estudo, nos dias de hoje a tecnologia tem avançado
muito para facilitar a inclusão de pessoas com deficiência, e assim tornar a área da informática
muito mais inclusiva. Podemos ver, como exemplo, o produto desenvolvido pela empresa TiX
Tecnologia Assistiva, chamado de Colibri \cite{tecassistivaColibriTIXTecnologia}, que tem como objetivo principal permitir
pessoas com certos tipos de deficiência como tetraplegia, paralisia cerebral e mesmo pessoas
com dificuldades motoras que sofreram algum tipo de acidente, como um AVC ou até automobilístico,
de utilizar o computador através de movimentos corporais.

A utilização do produto é feita por forma de movimentos corporais, por exemplo, piscando os
olhos para simular o clique do mouse, movimentando a cabeça para andar com mouse na tela e
até mesmo inclinando a cabeça para os lados fazendo a rolagem da página. 

Com isso, com a ajuda de um teclado virtual, podemos fazer todas as funções do conjunto de
mouse e teclado. O hardware é colocado em um óculos e é conectado com o bluetooth para qualquer
aparelho, tanto celular quanto computador. Podemos ver no vídeo de demonstração do produto
\cite{ColibriUseCelulares} que o óculos é de fácil manuseio, onde a pessoa que testa consegue acessar o
youtube e ver um vídeo nos primeiros minutos de uso.

Ao mesmo tempo que a tecnologia é inclusiva, ela restringe seu uso para um grupo social,
vide ao seu alto valor o qual pode ser cobrado mensalmente ou anualmente, cerca de cem
reais mensais como forma de aluguel, mas também comprando o produto em uma única parcela,
pagando dois mil novecentos e noventa reais.

Outra forma de se utilizar a tecnologia, essa muito mais inclusiva do ponto de vista econômico,
seria fazendo uma versão caseira, pois o software que é utilizado para confecção do produto é
de código-aberto, então, é possível construir em casa uma versão utilizando alguns componentes
eletrônicos e um arduino. Solução essa que ao mesmo tempo se mostra muito boa, precisa de uma
experiência e conhecimento em programação e eletrônica, ainda restringindo a parcela de pessoas
que conseguem usufruir.

Além disso, a discussão do campo da inclusão, e os modos para se alcançar a mesma, vai longe
pelas extensas demandas, ao mesmo tempo que perto quando pensamos em espaços físicos. A escola
é um exemplo disso. Estudantes com deficiência física têm dificuldades motoras que limitam
sua capacidade na forma regular de ensino sem adaptações. A falta dessas adaptações pode
impedir que estes estudantes comprometam seu processo de ensino aprendizagem.

Contudo, temos na atualidade o modelo de ensino remoto, que por meio da internet nivelou a
forma de ensino na qual todos têm acesso à aulas gravadas, por exemplo, possibilitando um
aluno com deficiência reassistir certas aulas, coisas que normalmente não poderiam ser feitas
pois o modelo das aulas presenciais pode vir a depender muito da anotação do aluno para melhor
fixação, fato esse que muitas vezes é impossibilitado devido a falta de recursos adaptados.

% ----------------------------------------------------------
% Finaliza a parte no bookmark do PDF
% para que se inicie o bookmark na raiz
% e adiciona espaço de parte no Sumário
% ----------------------------------------------------------
\phantompart
% ---
% Conclusão
% ---
\chapter{Conclusão}
Vide os fatos aqui discorridos neste estudo, temos que a acessibilidade a tecnologia para
pessoas com deficiência é um grande desafio que deve ser atacado por nós como sociedade.

A tecnologia e a informática fazem parte do cotidiano de todas as pessoas,
porém deficiêntes possuem certas dificuldade de acesso para tais tecnologias.
Cabe à área tecnológica pensar e melhorar como um todo para se tornar muito mais inclusiva,
melhorando a qualidade de vida dessas pessoas.

\postextual
% ----------------------------------------------------------
% Referências bibliográficas
% ----------------------------------------------------------
\printbibliography
%---------------------------------------------------------------------
% INDICE REMISSIVO
%---------------------------------------------------------------------
\phantompart
\printindex
%---------------------------------------------------------------------
\end{document}
