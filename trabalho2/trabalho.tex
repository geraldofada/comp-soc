\documentclass[
    12pt, % tamanho da fonte
    a4paper, % tamanho do papel.
    oneside, % impede a configuração para impressão frente e verso
    english, % idioma adicional para hifenização
    brazil % o último idioma é o principal do documento
]{abntex2}

% ---
% Pacotes básicos
% ---
\usepackage{lmodern}
\usepackage[T1]{fontenc} % Selecao de codigos de fonte.
\usepackage[utf8]{inputenc} % Codificacao do documento (conversão automática dos acentos)
\usepackage{csquotes}
\usepackage{indentfirst} % Indenta o primeiro parágrafo de cada seção.
\usepackage{color} % Controle das cores
\usepackage{graphicx} % Inclusão de gráficos
\usepackage{microtype} % para melhorias de justificação
% ---

% Símbolo ordinal
\usepackage{newunicodechar}
\newunicodechar{°}{\ensuremath{^\circ}}

% ---
% Pacotes de citações
% ---
\usepackage[style=abnt]{biblatex}
\addbibresource{referencias.bib}

% ---
% Informações de dados para CAPA e FOLHA DE ROSTO
% ---
\title{Vantagens do software proprietário e desvantagens do software livre \\ \small{Computação e Sociedade}}
\author{Bruno Mello\and Geraldo Fada\and Luciano Lage\and Rodrigo Correa}
\local{Brasil, Rio de Janeiro}
\date{10 de Agosto 2021\\ \small{1° Semestre}}
\orientador[Professor:]{José Raphael Bokehi}
\instituicao{Universidade Federal Fluminense}
\tipotrabalho{Trabalho Acadêmico}

% ---
% Configurações de aparência do PDF final
% ---
% alterando o aspecto da cor azul
\definecolor{blue}{RGB}{60,100,120}
% informações do PDF
\makeatletter
\hypersetup{
    %pagebackref=true,
    pdftitle={\@title},
    pdfauthor={\@author},
    pdfsubject={\imprimirpreambulo},
    pdfcreator={LaTeX with abnTeX2},
    pdfkeywords={abnt}{latex}{abntex}{abntex2}{trabalho acadêmico},
    colorlinks=true, % false: boxed links; true: colored links
    linkcolor=blue, % color of internal links
    citecolor=blue, % color of links to bibliography
    filecolor=magenta, % color of file links
    urlcolor=blue,
    bookmarksdepth=4
}
\makeatother
% ---

% ---
% Espaçamentos entre linhas e parágrafos
% ---
% O tamanho do parágrafo é dado por:
\setlength{\parindent}{1.3cm}
% Controle do espaçamento entre um parágrafo e outro:
\setlength{\parskip}{0.2cm}  % tente também \onelineskip

% ----
% Início do documento
% ----
\begin{document}

% Seleciona o idioma do documento (conforme pacotes do babel)
\selectlanguage{brazil}

% Retira espaço extra obsoleto entre as frases.
\frenchspacing

\imprimirfolhaderosto

% ---
% inserir o sumario
% ---
\pdfbookmark[0]{\contentsname}{toc}
\tableofcontents*
\cleardoublepage

\textual
\chapter{Introdução}


\chapter{Desenvolvimento}

\section{Software sem uma visão única não progride}
Existe um ponto negativo que afeta muito os softwares livres e é, infelizmente,
algo inerente à sua concepção. Quando algo é feito de forma colaborativa,
por definição, não existe uma hierarquia propriamente dita, pois é a contribuição
da comunidade que constrói esse algo e é nesse ponto que o software livre é
abalado em relação aos seus competidores proprietários: a falta de hierarquia.

A ideia de que um software é livre e que qualquer pessoa pode contribuir com
sua construção se desemboca em dois caminhos, ou o conceito de liberdade é
quebrado e não é realmente todo mundo que pode contribuir ou de fato existe
uma tentativa de aceitar todo e qualquer tipo de contribuição. De toda forma,
ambas situações afetam igualmente os projetos abertos. Existe também um
terceiro ramo, provavelmente o mais comum e o que gera as piores
consequências, que ocorre quando o primeiro caminho é tomado, mas devido as
suas consequências e sua grandeza no problema esse caso vai ser retomado por último.

Para avaliar o primeiro caso podemos usar o repositório do kernel Linux como
exemplo, que é software aberto de talvez maior importância atualmente. Sendo
responsável por mais de 40\% do tráfego na internet \cite{LinuxVsWindows}, pelos serviços AWS
da Amazon que silenciosamente rodam mais de um terço dos sites da web e usam
Linux por trás \cite{WhatAWSDoes2020,brandomUsingInternetAmazon2018,clarkHowAmazonWeb2014}
e além disso, também está por trás do Android que por
si só domina o mercado de celulares com mais 70\% \cite{101AndroidStats2019,burnettePatrickBradyDissects},
fica fácil notar a importância desse projeto para o mundo. O kernel Linux é
famoso por ter uma única pessoa que encabeça todo o projeto (Linus Torvalds),
aceitando apenas o que ele julga ser de excelência dos seus colaboradores
\cite{swannerLinuxCreatorLinus2015,osborneLinusTorvaldsDon}, esse simples fato
já enfatiza o ponto: o kernel Linux não é de fato aberto, visto que não é qualquer
um que pode modificá-lo.

Olhando por outra lado, o do sucesso que o kernel adquiriu, pode-se atribuir
grande parte desse sucesso pela hierarquia extremamente bem definida e
consequentemente a única visão que o software segue, a do seu criador. Dessa
forma temos como exemplo o maior ponto fora da curva, que exatamente por não
seguir a filosofia aberta, conseguiu seguir uma única visão e manter seu nível
de qualidade e o seu foco no que se propôs a fazer.

O segundo caso beira o impossível de achar casos, pois como já foi discutido
um projeto que bota em prática de verdade o conceito de aberto teria de aceitar
contribuições de qualquer pessoa e isso torna qualquer tipo de progressão
inviável. Basta pensar do ponto de vista de um engenheiro, que precisa construir
uma ponte, mas ao mesmo tempo existem milhares de pessoas trocando o tipo
de material que vai ser utilizado, redesenhando o projeto, mudando o destino
final da ponte, enfim, é simplesmente inviável. E o próprio fato de não ter
nenhum caso famoso aplicando ao pé da letra esse conceito já mostra que nenhum
projeto se sustentou de fato aplicando esse método.

Agora entrando no caso citado do segundo parágrafo que gera as piores consequências:
a fragmentação. Esse problema ocorre quando uma pessoa ou uma parte da comunidade
possui uma visão diferente de um projeto, e por não poder contribuir ou não concordar
com a versão vigente se cria uma nova. A partir dessa ideia surgiu o conceito do
fork, é criado uma bifurcação do projeto.

O Android é um dos maiores exemplos desse problema, por ser um projeto também open
source \cite{PlatformSuperprojectAndroid} as fabricantes de celulares criam versões
próprias para seus aparelhos,
a Samsung, Xiaomi, Google, LG, Huawei, Oppo, Motorola, Sony, Asus, OnePlus são alguns
exemplos de fabricantes que suportam apenas suas variantes de android e as mesmas
partem de versões específicas do android original, gerando uma fragmentação na
base usuário que, por exemplo, em 2015, 40\% das pessoas usavam uma versão muito
antiga do Android enquanto apenas 0,8\% usava a versão mais recente \cite{GraficoMostraComo2015}.

Enquanto isso na concorrência, a Apple lida com esse problema de forma muito melhor.
Entregando o iOS para seus usuários que é um produto fechado, com apenas celulares
da empresa utilizando-o, no mesmo ano que o Android possuia uma minúscula base
de 0,8\% com os celulares atualizados a Apple em menos de 44 horas já tinha
atualizado mais 20\% da sua base \cite{cfaAppleIOSPasses2015}.

Pode-se citar outro exemplo mais nichado, os compositores do desktop Linux.
Compositores são pedaços de tecnologia utilizados para fazer uma interface
entre o sistema de janelas do sistema operacional e a computação gráfica \cite{CompositorsLinux},
ou seja, é com ele que qualquer sistema operacional moderno ganha sua aparência:
janelas transparentes e com efeito de blur, animações, sombreados e o mais importante
o V-Sync que tira o bug visual de tearing (a tela parece ficar se dividindo
de tempos em tempos).

Um projeto que tinha como objetivo fazer exatamente essa camada entre a API
gráfica do computador com as janelas do sistema chamado Picom \cite{yshuiPicom2021}, sofreu
muito com a falta de uma visão única para seu projeto. Membros da comunidade
resolveram adicionar funcionalidades que não estavam disponíveis no projeto
oficial, levando a criação de mais outras três grandes versões do mesmo
software \cite{PicomStandaloneXorg}. Uma versão adicionou a capacidade de ter bordas redondas nas
janelas e o efeito de blur \cite{IbhagwanPicomLightweight}, outra melhorou o efeito de blur utilizando
um algorítmo mais performático \cite{Tryone144PicomFeature} e por fim a terceira tenta juntar ambas
features mas não consegue se manter sempre atualizada \cite{JonaburgPicomLightweight} pois falta
contribuintes e os donos originais não abandonam suas visões do projeto
o que os fazem se manter com suas próprias versões. 

Além de gerar inconsistências entre os projetos, os usuários que resolvem
utilizar esses compositores se prejudicam pois precisam decidir entre
funcionalidades que poderiam facilmente coexistir. Também é prejudicial para
as empresas por trás das distribuições do desktop Linux que precisam manter
essas funcionalidades e acabam por criar ainda mais uma versão nova para
o seu projeto \cite{kochtaLinuxDesktopCompositors2018}.

É possível perceber como a falta de uma visão única não beneficia nenhum
projeto na comunidade aberta e ainda sim esse é um problema que não tem
como fugir, pois o próprio conceito de software livre cria essas fragmentações.

\section{Não atende bem um mercado específico}
O mercado de tecnologia vem crescendo cada vez mais, e move uma grande
parte da economia de alguns países, isso se dá ao fato de que atender
uma demanda de nicho é algo muito mais lucrativo do que ter um projeto
mais abrangente. Quando esse projeto abrangente é aberto pra todo mundo
poder opinar, dificilmente conseguimos monetizar uma certa mudança e
ganhar dinheiro com isso, e além disso, como todos possuem acesso àquela
informação, dificilmente ela progride facilmente para fazer o que queremos
rapidamente.

Podemos olhar como indicativo disso o mercado empresarial, que contrata
softwares por demanda para realizar certas tarefas específicas ou até
mesmo gerenciar os recursos internos da firma, qualquer um pode fazer
isso, desde um consultório ortodôntico até grandes conglomerados de
tecnologia da informação. Existem empresas que, assim como descritas na seguinte
matéria \cite{XAppsSoftwareSob}, podem ser consideradas “fábricas de apps”, que
geralmente desenvolvem um programa para fazer um certo processo e que,
depois daquele esqueleto construído, fazem mudanças específicas para suprir
a necessidade da empresa contratante.

Se todos tem acesso ao esqueleto, ninguém pagaria esse esqueleto e sim
somente a mudança, se tornando muito menos rentável, sendo que a parte
mais difícil é em si construir aquela base, então pode haver até a
possibilidade da empresa que queria aquele tal software, como ela já
possui uma estrutura inicial, construir de maneira própria uma versão
que faz o que ela quer, e sem o incentivo financeiro, muitas vezes ninguém
vai fazer construir tal esqueleto.

Outro ponto que acaba influenciando é que quando há algum problema naquela
aplicação, muitas vezes quem utiliza aquele software, no caso de algum
problema, acaba querendo procurar um suporte técnico para o produto, algo
que dificilmente vai ser encontrado com o open source. Até pelo fato de que
muitas vezes a utilização é pouco intuitiva para o usuário, arrumar problemas
é ainda mais complicado. Muitas vezes as empresas contratantes até pagam para
ter o suporte da aplicação, tornando ainda mais lucrativo e movendo ainda
mais dinheiro. 

Pode-se ver isso até mesmo na comunidade open source, em que as empresas que mandam
são as gigantes como a Red-Hat que fornecem distribuições linux de forma
empresarial e cobrando, a Canonical também começou a seguir por esse lado
com versões específicas do Ubuntu, e isso só funciona pois, mais do que o
sistema operacional fazendo o quê o contratante deseja, a empresa presta o
serviço de suporte, e ainda fugindo completamente da ideia de código aberto
pois essas versões acabam sendo proprietárias. 

\section{Pouco intuitivo para o usuário final}
Uma das características que desfavorece a utilização de softwares open source
é que muitos projetos abertos só estão disponíveis para uma pequena parcela
de pessoas que tenha o conhecimento técnico necessário para utilizar o programa,
existem casos onde o usuário precisa compilar o programa diretamente da fonte. 

Como exemplo para o caso, o integrante do grupo Bruno Mello utiliza o software
livre LogiOps, o mesmo é utilizado para sistemas operacionais Linux, o software
tem como objetivo mapear funções dos mouses da LogiTech, pois a empresa apenas
desenvolveu software proprietário para Windows, deixando usuários Linux sem
suporte para utilizar seus mouses, dessa forma a própria comunidade se uniu
para desenvolver o LogiOps \cite{garciaexplicaLogitechOptionsPara}.
O grande problema identificado pelo Bruno
foi o fato de ser preciso compilar o código fonte do software via prompt de comando.

Seguindo por essa linha de raciocínio existem também problemas relacionados a
alguns softwares open source serem pouco intuitivos para o usuário final, pois
nem sempre o usuário tem os conhecimentos necessários para resolver determinados
problemas que possam vir a ocorrer, e não necessariamente terão fácil acesso
as respostas para esses problemas. 

Podemos utilizar como exemplo a Valve que está desenvolvendo uma funcionalidade
na Steam para executar jogos em Linux que não foram desenvolvidos propriamente
para esse tipo de sistema operacional. Realizando uma rápida busca pela internet,
podemos facilmente achar tutoriais extremamente longos e complexos como o
vídeo “Como Jogar no Linux em 2020 - Guia Completo (+Dicas Avançadas)” do Youtuber
Diolinux \cite{diolinuxComoJogarNo}. Existem diversos relatos de usuários que tiveram problemas com
essa funcionalidade e que para poder jogar os seus jogos e se divertir um pouco,
deveriam executar scripts manuais via prompt de comando, que como já informado
anteriormente, não são todos os usuários que possuem esse conhecimento para seguir.

Parte desses problemas relacionados a utilização do usuário final se dá em função
da maioria dos softwares open source ser mantida quase que única e exclusivamente
por equipes pequenas ou pela comunidade, dessa forma não é possível atender toda
a demanda que o mercado precisa, pois não existe mão de obra para testar tantas
possibilidades de sistemas operacionais e hardwares diferente, o que acaba gerando
problemas de compatibilidade com diversas pessoas, o que acaba gerando mais demanda
para correção e consulta, que muitas vezes não tem resposta de imediato. Esse
problema acaba se tornando uma bola de neve, pois não é possível dispor de uma força
tarefa para atuar em tantos campos, e sem visão econômica para o projeto em si, é
mais difícil ainda investir para aumentar a equipe, que muitas vezes trabalha no “amor”.

Assim como os problemas relacionados a compatibilidade, muitas vezes não existe
“braço” o suficiente para atuar em uma interface gráfica intuitiva para o usuário,
e muitos programas rodam sua totalidade via prompt de comando ou com pouca
interação gráfica. Utilizando como exemplo o VLC Player utiliza uma interface
gráfica mais bruta e rustica, em tempos onde o design da interface tem tornado-se
cada vez mais simplificado, intuitivo e moderno. Navegando pelos fóruns da internet,
é possível observar que diversos usuários do VLC Player em linux, apresentam
constantes problemas em decorrência de sua interface, que só são resolvidos muitas
vezes recompilando o software, exemplo disso temos o tópico em um fórum direcionado
para o sistema operacional UBUNTU, onde um usuário tem o seu VLC simplesmente sem
interface alguma, onde precisa seguir passos via prompt de comando
para poder corrigir \cite{VLCMediaPlayer}. 

Todos esses pontos acabam levando a poucos usuários que realmente tenham conhecimento
para utilizar esses software, dessa forma a evolução dos mesmos fica estagnada,
pois não ter público equivale a não ter sucesso no mercado, sem sucesso no
mercado não existe avanço do projeto.


\section{Dificuldade de empacotamento}
Outro ponto de desvantagem do software open source, como explica Linus Torvalds em
uma DebConf, \cite{gentoomanLinusTorvaldsWhy} dessa vez focando especificamente no empacotamento e
distribuição de programas, é a existência de diversas variações de sistemas operacionais
baseados em linux, fazendo com que seja necessária a compilação e manutenção de
binários para cada uma dessas versões. Um desenvolvedor de uma aplicação consegue
facilmente criar uma compilação do seu programa para windows, e essa mesma compilação
será utilizada por todos os usuários desse sistema, no máximo sendo necessária a
distinção entre uma compilação de 32 bits e outra de 64 bits. O mesmo pode ser dito
para aplicações geradas para MacOS: O desenvolvedor só precisa se preocupar em compilar
para esta plataforma. No caso de linux isso é um pouco diferente, existem sistemas
baseados em debian, arch, bsd, e mesmo sistemas de mesma base possuem peculiaridades.
Ao compilar para debian stable, por exemplo, é necessário que o programa esteja de
acordo com certas regras de uso de bibliotecas compartilhadas para que esse programa
seja aceito nos repositórios da distribuição.

"Making binaries for linux desktop applications is a major fucking pain in the ass"
- Linus Torvalds


% ----------------------------------------------------------
% Finaliza a parte no bookmark do PDF
% para que se inicie o bookmark na raiz
% e adiciona espaço de parte no Sumário
% ----------------------------------------------------------
\phantompart
% ---
% Conclusão
% ---
\chapter{Conclusão}

\postextual
% ----------------------------------------------------------
% Referências bibliográficas
% ----------------------------------------------------------
\printbibliography
%---------------------------------------------------------------------
% INDICE REMISSIVO
%---------------------------------------------------------------------
\phantompart
\printindex
%---------------------------------------------------------------------
\end{document}
